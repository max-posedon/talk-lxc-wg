\documentclass{beamer}
\usepackage[english,russian]{babel}
\usepackage[utf8]{inputenc}
\usepackage[T1]{fontenc}
\usepackage{graphicx}
\usepackage{ulem}

\mode<presentation>
{
    \usetheme{Warsaw}
    \useoutertheme{miniframes}
    \setbeamercovered{transparent}
}

\begin{document}

\title{How Wargaming.net use LXC \\for staging environments}
\author{Maksim 'max\_posedon' Melnikau}
\institute{Linux Mobile hobbyist\\World of Tanks developer}
\date{\today}
\frame{\titlepage}

\section{Overview}
\frame{
    \frametitle{LXC}
    \begin{block}{Wikipedia}
    LXC (Linux Containers) is an operating system-level virtualization 
    method for running multiple isolated Linux systems (containers) 
    on a single control host. LXC does not provide a virtual machine, 
    but rather provides a virtual environment that has its own process and network space.
    \\
    LXC relies on the Linux kernel cgroups functionality that became 
    available in version 2.6.29, developed as part of LXC. It also relies 
    on other kinds of namespace-isolation functionality, which were 
    developed and integrated into the mainline Linux kernel.
    \end{block}
}

\frame{
    \frametitle{Features}
    \begin{itemize}
    \item Lightweight - chroot on steroids
    \item Mainline kernel - available everywhere
    \item Can coexist with other virtualization technologies
    \item Effective - only one kernel schedule resources
    \end{itemize}
}

\section{Game Server staging}
\frame{
    \frametitle{Problems}
    \begin{block}{Games}
    \begin{itemize}
    \item World of Tanks
    \item World of Warplanes
    \end{itemize}
    \end{block}

    \begin{block}{Multipling problems}
    \begin{itemize}
    \item About 10 instances for testers
    \item Separate env for every server developer
    \item Fast and easy deployment required
    \item Deployment not automated
    \end{itemize}
    \end{block}
}

\frame{
    \frametitle{Solution}
    \begin{block}{Copy prepared container}
    \begin{itemize}
    \item Ubuntu Server 11.10 as host
    \item Centos 5.7 as guest
    \item All configs are ip and hostname agnostic
    \item Pass mac-address and hostname from lxc config
    \item Use office DNS/DHCP infrastructure
    \item Bind rabbitmq's mnesia storage to localhost, not hostname
    \item 2 network interfaces - external and internal
    \end{itemize}
    \end{block}
}

\section{WEB staging}
\frame{
    \frametitle{Problems}
    \begin{block}{Projects}
    \begin{itemize}
    \item SPA - single point of authentication
    \item Portal - game portal
    \item Exporter - export data form server
    \item CSIS - cluster state information service
    \item Update Service - news and patches info for launcher
    \item ...
    \end{itemize}
    \end{block}

    \begin{block}{Multipling problems}
    \begin{itemize}
    \item Multiple games
    \item Multiple regions
    \item Stable/Trunk/Experimental
    \end{itemize}
    \end{block}
}

\frame{
    \frametitle{Solution}
    \begin{itemize}
    \item Ubuntu Server 11.10 as host
    \item CentOS 5.7 -> Ubuntu Server 11.10 as guest
    \item Copy base CentOS container (OpenVZ's)
    \item Separate lxc container per project instance
    \item Use lxc template for installing Ubuntu
    \item Use puppet for automatic preparing system
    \item Use fabric for automatic deploying app
    \item Separate management user "with crontab"
    \end{itemize}
}

\section{}
\frame{
    \frametitle{Future Goals}
    \begin{itemize}
    \item Improve dencity of containers per hardware
    \item Prepare "all in" virtualbox image
    \end{itemize}
}

\frame{
    \frametitle{Questions and Answers}
    \begin{itemize}
    \item email/jabber(xmpp): m\_melnikau@wargaming.net
    \end{itemize}
}

\end{document}

